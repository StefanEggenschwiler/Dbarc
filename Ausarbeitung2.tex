\documentclass[10pt]{scrreprt}

%Math
\usepackage{amsmath}
\usepackage{amsfonts}
\usepackage{amssymb}
\usepackage{amsthm}
\usepackage{ulem}
\usepackage{stmaryrd} %f\UTF{00FC}r Blitz!

%PageStyle
\usepackage[ngerman]{babel} % deutsche Silbentrennung
\usepackage[utf8]{inputenc}
\usepackage{fancyhdr, graphicx}
\usepackage[scaled=0.92]{helvet}
\usepackage{enumitem}
\usepackage{parskip}
\usepackage[a4paper,top=2cm]{geometry}
\usepackage{framed}
\setlength{\textwidth}{17cm}
\setlength{\oddsidemargin}{-0.5cm}

% Shortcommands
\newcommand{\Bold}[1]{\textbf{#1}} %Boldface
\newcommand{\Kursiv}[1]{\textit{#1}} %Italic
\newcommand{\T}[1]{\text{#1}} %Textmode
\newcommand{\Nicht}[1]{\T{\sout{$ #1 $}}} %Streicht Shit durch

%Arrows
\newcommand{\lra}{\leftrightarrow}
\newcommand{\ra}{\rightarrow}
\newcommand{\la}{\leftarrow}
\newcommand{\lral}{\longleftrightarrow}
\newcommand{\ral}{\longrightarrow}
\newcommand{\lal}{\longleftarrow}
\newcommand{\Lra}{\Leftrightarrow}
\newcommand{\Ra}{\Rightarrow}
\newcommand{\La}{\Leftarrow}
\newcommand{\Lral}{\Longleftrightarrow}
\newcommand{\Ral}{\Longrightarrow}
\newcommand{\Lal}{\Longleftarrow}

% Code listenings
\usepackage{color}
\usepackage{xcolor}
\usepackage{listings}
\usepackage{caption}
\DeclareCaptionFont{white}{\color{black}}
\DeclareCaptionFormat{listing}{\colorbox{white}{\parbox{\textwidth}{#1#2#3}}}
\captionsetup[lstlisting]{format=listing,labelfont=white,textfont=white,font=bf}


\lstdefinestyle{sqlNoTitle}{
   language=SQL,
   basicstyle=\footnotesize\ttfamily, % Standardschrift
   backgroundcolor=\color[RGB]{255,255,0}, % Hintergrundfarbe
   numbers=left, % Ort der Zeilennummern
   numberstyle=\tiny, % Stil der Zeilennummern
   stepnumber=1, % Abstand zwischen den Zeilennummern
   numbersep=5pt, % Abstand der Nummern zum Text
   tabsize=2, % Groesse von Tabs
   extendedchars=true, %
   breaklines=true, % Zeilen werden Umgebrochen
   frame=trbl, % Rahmen
   stringstyle=\color[RGB]{42,0,255} \ttfamily, % Farbe der String
   keywordstyle=\color[RGB]{127,0,85} \bfseries, % Farbe der Keywords
   commentstyle=\color[RGB]{63,127,95} \ttfamily, % Farbe des Kommentars
   showspaces=false, % Leerzeichen anzeigen ?
   showtabs=false, % Tabs anzeigen ?
   xleftmargin=17pt,
   framexleftmargin=17pt,
   framexrightmargin=5pt,
   framexbottommargin=5pt,
   framextopmargin=5pt,
   showstringspaces=false, % Leerzeichen in Strings anzeigen ?
}


\lstdefinestyle{sql}{
   style=sqlNoTitle,
   % title=SQL-Query
}

\lstdefinestyle{queryexecutionplan}{
  basicstyle=\scriptsize\ttfamily, % Standardschrift
  backgroundcolor=\color[RGB]{238,233,233}, % Hintergrundfarbe
  frame=trbl, % Rahmen
  %title=Ausführungsplan
}

%Config
\renewcommand{\headrulewidth}{0pt}
\setlength{\headheight}{15.2pt}

%Metadata
\fancyfoot[C]{}
\title{
\vspace{4cm}
\huge{Datenbank Architektur für Fortgeschrittene}\\
\vspace{0.2cm}
\Large{Ausarbeitung 2: Zugriffssteuerung und Views}\\
}
\author{Daniel Gürber \cr Stefan Eggenschwiler}
\date{21.06.2013}

% hier beginnt das Dokument
\begin{document}

% Titelbild
\maketitle
\thispagestyle{fancy}

\newpage

% Inhaltsverzeichnis
\pagenumbering{Roman}
\tableofcontents	


\newpage
\setcounter{page}{1}
\pagenumbering{arabic}

% Inhalt Start
\chapter{Oracle 11g}
\section{Vorbereitung}
\subsection{Einrichten User}
\begin{lstlisting}[style=sql]
CREATE USER nutzer01 IDENTIFIED BY nutzer01;
CREATE USER nutzer02 IDENTIFIED BY nutzer02;

GRANT CREATE SESSION TO nutzer01;
GRANT CREATE SESSION TO nutzer02;

GRANT CREATE ROLE TO scott;
GRANT CREATE VIEW TO scott;
\end{lstlisting}

\section{Zugriffssteuerung mit User und Rollen}
\subsection{Tabellen erzeugen}
\begin{lstlisting}[style=sql]
DROP TABLE klassen;
DROP TABLE studenten;

CREATE TABLE klassen(
k_id number(9),
k_bezeichnung VARCHAR2(20),
k_zimmer VARCHAR2(10),
k_server VARCHAR2(10));

CREATE TABLE studenten(
s_id number(9),
s_name VARCHAR2(10),
s_vname VARCHAR2(10),
s_tel VARCHAR2(20),
s_konto_stand NUMBER(9 ),
s_klasse NUMBER(9));


INSERT INTO klassen values ( 10, 'ia00', '3333' , 'pluto');
INSERT INTO klassen values ( 20, 'ia01', '2222' , 'saturn');

INSERT INTO studenten values( 101, 'meier', 'hans', '11111', 5000, 10);
INSERT INTO studenten values( 102, 'hirt', 'otto', '22222', -100, 10);
INSERT INTO studenten values( 103, 'kok', 'thomas', '33333', 1000, 20);
INSERT INTO studenten values( 104, 'guzman', 'anna', '44444', 3000, 20);
INSERT INTO studenten values( 105, 'lorch', 'felix', '45678', 7000, 20);
\end{lstlisting}
\subsection{View erstellen}
\begin{lstlisting}[style=sql]
CREATE VIEW view1 AS
select s_name, s_vname, k_zimmer, k_server
FROM studenten, klassen
WHERE s_klasse = k_id
\end{lstlisting}

\subsection{Rollen definieren}
\begin{lstlisting}[style=sql]
CREATE ROLE view1_verwalter;
CREATE ROLE view1_nutzer;
\end{lstlisting}

\subsection{Den Rollen Rechte zuweisen}
\begin{lstlisting}[style=sql]
GRANT INSERT,SELECT,UPDATE,DELETE ON studenten TO view1_verwalter;
GRANT INSERT,SELECT,UPDATE,DELETE ON klassen TO view1_verwalter;
\end{lstlisting}

\begin{lstlisting}[style=sql]
GRANT SELECT ON view1 TO view1_nutzer;
\end{lstlisting}

\subsection{Den User Rollen zuweisen}
\begin{lstlisting}[style=sql]
GRANT view1_verwalter TO nutzer01;
GRANT view1_nutzer TO nutzer02;
\end{lstlisting}

\subsection{Überprüfung der Rechte der beiden Rollen}
\subsubsection{Was kann der Verwalter lesen und bearbeiten?}
\begin{lstlisting}[style=sql]
SELECT * FROM scott.studenten;
INSERT INTO scott.studenten VALUES (106, 'peter', 'muster', 12345,2000,20);
UPDATE scott.studenten SET s_tel = 54321 WHERE s_id = 106;
DELETE FROM scott.studenten WHERE s_id = 106;

SELECT * FROM scott.klassen;
INSERT INTO scott.klassen VALUES (30, 'ia02', 1111, 'uranus');
UPDATE scott.klassen SET k_zimmer = 4444 WHERE k_id = 30;
DELETE FROM scott.klassen WHERE k_id = 30;

SELECT * FROM scott.view1;
INSERT INTO scott.view1 VALUES ('peter','muster',2222,'saturn');
UPDATE scott.view1 SET s_name = 'kook' WHERE s_vname = 'thomas';
DELETE FROM scott.view1 WHERE s_vname = 'thomas';
\end{lstlisting}
ORA-00942: Tabelle oder View nicht vorhanden
\subsubsection{Was kann der Nutzer lesen und bearbeiten?}
ORA-00942: Tabelle oder View nicht vorhanden
ORA-01031: Nicht ausreichende Berechtigungen
\section{Zugriffsrechte: Objekt- und Systemrechte}
\subsection{Objektrechte}
Als scott:
\begin{lstlisting}[style=sql]
CREATE VIEW view2 AS
select s_name, s_vname, k_zimmer, k_server
FROM studenten, klassen
WHERE s_klasse = k_id;

GRANT SELECT ON view2 TO nutzer01 WITH GRANT OPTION;
\end{lstlisting}

Als nutzer01:
\begin{lstlisting}[style=sql]
GRANT SELECT ON scott view2 TO nutzer02;
\end{lstlisting}
Erfolgreich.

Als nutzer02:
\begin{lstlisting}[style=sql]
SELECT * FROM scott view2;
\end{lstlisting}
Erfolgreich.

Als scott:
\begin{lstlisting}[style=sql]
REVOKE SELECT ON view2 FROM nutzer01;
\end{lstlisting}

Als nutzer02:
\begin{lstlisting}[style=sql]
SELECT * FROM scott view2;
\end{lstlisting}
ORA-00942: Tabelle oder View nicht vorhanden

\subsection{Systemrechte}
Als system:
\begin{lstlisting}[style=sql]
REVOKE create session FROM nutzer01;
REVOKE create session FROM nutzer02;
\end{lstlisting}
ORA-01045: user STUDENT02 lacks CREATE SESSION privilege; logon denied.

Als system:
\begin{lstlisting}[style=sql]
GRANT create session TO nutzer01 WITH ADMIN OPTION;
\end{lstlisting}

als nutzer01:
\begin{lstlisting}[style=sql]
GRANT create session TO nutzer02;
\end{lstlisting}
Beide können Sessions erstellen.

Als system:
\begin{lstlisting}[style=sql]
REVOKE create session FROM nutzer01;
\end{lstlisting}
nutzer02 kann IMMERNOCH Sessions erstellen. (Keine Kaskade)

\section{Views}
\subsection{Rechte von Views}
als system:
\begin{lstlisting}[style=sql]
GRANT CREATE VIEW TO nutzer01;
\end{lstlisting}

als nutzer01:
\begin{lstlisting}[style=sql]
CREATE VIEW cheat_view AS
SELECT * FROM scott.studenten
\end{lstlisting}
ORA-01031: Nicht ausreichende Berechtigungen

als system:
\begin{lstlisting}[style=sql]
GRANT CREATE ANY VIEW TO nutzer01;
\end{lstlisting}

als nutzer01:
ORA-01031: Nicht ausreichende Berechtigungen

als system:
\begin{lstlisting}[style=sql]
GRANT INSERT,SELECT,UPDATE,DELETE ON scott.studenten TO nutzer01;
\end{lstlisting}

als nutzer01:
erfolgreich

als nutzer01:
\begin{lstlisting}[style=sql]
GRANT SELECT ON cheat_view TO nutzer02;
\end{lstlisting}

als nutzer02:
\begin{lstlisting}[style=sql]
CREATE VIEW cheat_view AS
SELECT * FROM scott.studenten
\end{lstlisting}
ORA-01720: Berechtigungsoptionen für 'SCOTT.STUDENTEN' nicht vorhanden

\subsection{DDL-Änderungen an den Basistabellen}
als scott:
\begin{lstlisting}[style=sql]
ALTER TABLE studenten ADD birthday TIMESTAMP;
\end{lstlisting}

\begin{lstlisting}[style=sql]
SELECT * FROM view1;
\end{lstlisting}
Kein Problem, da view1 nicht beeinträchtigt wird.

\begin{lstlisting}[style=sql]
ALTER TABLE studenten DROP COLUMN birthday;
\end{lstlisting}

\begin{lstlisting}[style=sql]
ALTER TABLE studenten DROP COLUMN s_name;
\end{lstlisting}

\begin{lstlisting}[style=sql]
SELECT * FROM view1;
\end{lstlisting}
ORA-04063: view "view1" enthält Fehler

\begin{lstlisting}[style=sql]
ALTER TABLE studenten ADD s_name VARCHAR2(10);
\end{lstlisting}
Fixed.

\subsection{Updatable Views}
als scott:
\begin{lstlisting}[style=sql]
CREATE VIEW view_distinct AS
SELECT DISTINCT s_vname
FROM studenten;

UPDATE view_distinct SET s_vname='ana' WHERE s_vname='anna';
\end{lstlisting}
ORA-01732: Datenmanipulationsoperation auf dieser View nicht zulässig

\begin{lstlisting}[style=sql]
CREATE VIEW view_all AS
SELECT *
FROM studenten
INNER JOIN klassen ON s_klassen=k_id;
\end{lstlisting}

\begin{lstlisting}[style=sql]
SELECT * FROM view_all;

UPDATE view_all SET s_name='muster' WHERE s_id=101;
\end{lstlisting}
ORA-01779: Kann keine Spalte, die einer Basistabelle zugewiesen wird, verändern

\begin{lstlisting}[style=sql]
ALTER TABLE studenten
ADD CONSTRAINT STUDENT_PK PRIMARY KEY ( S_ID ) ENABLE;

ALTER TABLE klassen
ADD CONSTRAINT KLASSE_PK PRIMARY KEY ( K_ID ) ENABLE;
\end{lstlisting}

\begin{lstlisting}[style=sql]
UPDATE view_all SET s_name='muster' WHERE s_id=101;
\end{lstlisting}
erfolgreich aktualisiert.

\begin{lstlisting}[style=sql]
UPDATE view_all SET k_bezeichnung='test' WHERE k_id=10;
\end{lstlisting}
ORA-01779: Kann keine Spalte, die einer Basistabelle zugewiesen wird, verändern

\subsection{WITH CHECK OPTION}
\begin{lstlisting}[style=sql]
CREATE VIEW v1 AS
SELECT *
FROM studenten
WHERE s_klasse = 20 WITH CHECK OPTION;
\end{lstlisting}

\begin{lstlisting}[style=sql]
CREATE VIEW v2 AS
SELECT *
FROM v1
WHERE s_tel < 40000
\end{lstlisting}

\begin{lstlisting}[style=sql]
INSER INTO v2 VALUES (200, 'muster', 'peter', 40001, 2000, 20);
INSER INTO v2 VALUES (201, 'muster', 'peter', 40001, 2000, 10);
\end{lstlisting}
ORA-01402: Verletzung der WHERE-Klausel einer View WITH CHECK OPTION

\begin{lstlisting}[style=sql]
CREATE VIEW v3 AS
SELECT *
FROM v2
WHERE s_vname='peter' WITH CHECK OPTION;
\end{lstlisting}

\begin{lstlisting}[style=sql]
INSER INTO v3 VALUES (202, 'muster', 'thomas', 30000, 2000, 20);
\end{lstlisting}
ORA-01402: Verletzung der WHERE-Klausel einer View WITH CHECK OPTION

\begin{lstlisting}[style=sql]
INSER INTO v3 VALUES (202, 'muster', 'peter', 30000, 2000, 20);
\end{lstlisting}
Funktioniert.

\begin{lstlisting}[style=sql]
INSER INTO v3 VALUES (203, 'muster', 'thomas', 30000, 2000, 10);
\end{lstlisting}
ORA-01402: Verletzung der WHERE-Klausel einer View WITH CHECK OPTION

\begin{lstlisting}[style=sql]
INSER INTO v3 VALUES (204, 'muster', 'thomas', 40000, 2000, 20);
\end{lstlisting}
ORA-01402: Verletzung der WHERE-Klausel einer View WITH CHECK OPTION
% Inhalt Ende
\end{document} 
